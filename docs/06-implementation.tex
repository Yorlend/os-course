\chapter{Технологический раздел}

\section{Выбор языка и среды разработки}

Для написания драйвера был выбран язык программирования C.

Для сборки драйвера будет использована утилита Make \cite{toolbox}. В качестве среды разработки был выбран редактор VSCode \cite{vscode}.

Драйвер разрабатывается для ОС Linux версии 6.7.

\section{Точки входа драйвера}

\begin{lstlisting}[caption={функция \texttt{probe} драйвера}]
int fprint_probe(struct usb_interface *intf, const struct usb_device_id *id)
{
    printk(KERN_INFO LOG_PREFIX "fingerprint usb interface probed\n");

    struct usb_host_interface *iface = intf->cur_altsetting;
    printk(KERN_INFO LOG_PREFIX "fingerprint usb interface number: %d\n", iface->desc.bInterfaceNumber);
    if (iface->string != NULL) {
        printk(KERN_INFO LOG_PREFIX "fingerprint usb interface string: %s\n", iface->string);
    } else {
        printk(KERN_WARNING LOG_PREFIX "fingerprint usb interface string is NULL\n");
    }

    if (iface->desc.bNumEndpoints != 3) {
        printk(KERN_INFO LOG_PREFIX "skipping this interface\n");
        return 0;
    }

    if (iface->desc.bInterfaceNumber != 0) {
        printk(KERN_INFO LOG_PREFIX "skipping this interface\n");
        return 0;
    }

    struct usb_endpoint_descriptor *endpoint = &iface->endpoint[0].desc;
    if (endpoint->bEndpointAddress != 0x82) {
        printk(KERN_WARNING LOG_PREFIX "endpoint[1] must have adress 0x81\n");
        return 0;
    }

    int res = initialize_fprint_endpoint(intf, endpoint);
    if (res != 0) {
        printk(KERN_ERR LOG_PREFIX "failed to initialize endpoint\n");
    }
    return res;
}
\end{lstlisting}

\begin{lstlisting}[caption={функция \texttt{disconnect} драйвера}]
void fprint_disconnect(struct usb_interface *intf)
{
    printk(KERN_INFO LOG_PREFIX "fingerprint usb interface disconnected\n");

    struct usb_host_interface *iface = intf->cur_altsetting;
    printk(KERN_INFO LOG_PREFIX "fingerprint usb interface number: %d\n", iface->desc.bInterfaceNumber);
    printk(KERN_INFO LOG_PREFIX "fingerprint usb interface string: %d\n", iface->desc.iInterface);

    if (iface->desc.bInterfaceNumber != 0) {
        printk(KERN_INFO LOG_PREFIX "skipping this interface\n");
        return;
    }

    struct usb_endpoint_descriptor *endpoint = &iface->endpoint[0].desc;
    if (endpoint->bEndpointAddress != 0x82) {
        printk(KERN_WARNING LOG_PREFIX "endpoint[1] must have adress 0x81\n");
        return;
    }

    struct usb_device* usbdev = interface_to_usbdev(intf);
    struct fprint_drv_data* drvdata = dev_get_drvdata(&usbdev->dev);
    if (drvdata != NULL) {
        kfree(drvdata->transfer_buffer);
        kfree(drvdata);
        printk(KERN_INFO LOG_PREFIX "driver data free'd\n");
    }
}
\end{lstlisting}

\clearpage

\section{Работа с файловой системой proc}

\begin{lstlisting}[caption={функции работы с VFS proc}]
static atomic_t reader_available = ATOMIC_INIT(1);
static DECLARE_WAIT_QUEUE_HEAD(reader_queue);
static DECLARE_WAIT_QUEUE_HEAD(poll_wait_queue);

static const struct seq_operations fprint_seq_ops = {
    .start = start,
    .next = next,
    .stop = stop,
    .show = show
};

static const struct proc_ops fprint_proc_ops = {
    .proc_open    = open,
    .proc_read    = seq_read,
    .proc_lseek   = seq_lseek,
    .proc_release = seq_release,
    .proc_poll    = poll,
};

void* start(struct seq_file *m, loff_t *pos)
{
    seqfile_iterator = 0;
    return &seqfile_iterator;
}

void stop(struct seq_file *m, void *v)
{
    atomic_set(&reader_available, 1);
}

void* next(struct seq_file *m, void *v, loff_t *pos)
{
    wait_event_interruptible(reader_queue, signature_data_available);
    signature_data_available = false;

    *pos += signature_data_size;
    return NULL;
}

int show(struct seq_file *m, void *v)
{
    if (signature_data == NULL) {
        printk(KERN_WARNING LOG_PREFIX "signature data is NULL\n");
        return 0;
    }

    seq_write(m, signature_data, signature_data_size);
    return 0;
}

int open(struct inode* inode, struct file* file)
{
    signature_data_available = false;
    return seq_open(file, &fprint_seq_ops);
}

static unsigned int poll(struct file *file, poll_table *wait)
{
    poll_wait(file, &poll_wait_queue, wait);
    if (signature_data_available)
        return POLLIN | POLLRDNORM;
    return 0;
}
\end{lstlisting}

\section{Инициализация TLS-соединения}

\begin{lstlisting}[caption={Инициализация TLS-соединения}]
#define PSK_LENGTH 1024
static unsigned char global_psk[PSK_LENGTH];
static int is_first_attempt = 1;

struct conn_request {
    __u8 type;
#define CONNECTION_TLS_REQUEST 0x03
    __u8 data;
};

static int send_connection_request(struct usb_device *dev) {
    struct urb *urb = usb_alloc_urb(0, GFP_KERNEL);
    struct conn_request buffer = { .type = CONNECTION_TLS_REQUEST, .data = 0 };

    usb_fill_control_urb(urb, dev, usb_sndctrlpipe(dev, 0), &buffer, sizeof(struct conn_request), usb_callback_fn, NULL);
    usb_submit_urb(urb);
    return usb_wait_fn(urb);
}

static int get_psk(struct usb_device *dev, unsigned char *psk) {
    struct urb *urb = usb_alloc_urb(0, GFP_KERNEL);
    
    usb_fill_bulk_urb(urb, dev, usb_rcvbulkpipe(dev, 1), psk, PSK_LENGTH, usb_callback_fn, NULL);
    usb_submit_urb(urb);
    return usb_wait_fn(urb);
}

static int initialize_tls_connection(struct usb_device *dev) {
    int ret;
    unsigned char psk[PSK_LENGTH];

    ret = send_connection_request(dev);
    if (ret < 0) {
        printk(KERN_ERR LOG_PREFIX "Failed to send connection request to the fingerprint scanner\n");
        return ret;
    }

    ret = get_psk(dev, psk);
    if (ret < 0) {
        printk(KERN_ERR LOG_PREFIX "Failed to get PSK from the fingerprint scanner\n");
        return ret;
    }

    if (is_first_attempt) {
        memcpy(global_psk, psk, PSK_LENGTH);
        is_first_attempt = 0;
    } else {
        if (memcmp(global_psk, psk, PSK_LENGTH) != 0) {
            printk(KERN_ERR LOG_PREFIX "Mismatched PSK values, terminating TLS connection initialization\n");
            return -EINVAL;
        }
    }

    return 0;
}
\end{lstlisting}
